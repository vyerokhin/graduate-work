
%% LyX 2.1.3 created this file.  For more info, see http://www.lyx.org/.
%% Do not edit unless you really know what you are doing.
\documentclass[12pt,oneside,english,reqno]{amsbook}
\renewcommand{\familydefault}{\rmdefault}
\usepackage[T1]{fontenc}

\usepackage{geometry}
\geometry{verbose,tmargin=1in,bmargin=1in,lmargin=1in,rmargin=1in}
\usepackage{mathrsfs}
\usepackage{url}
\usepackage{amsbsy,latexsym,amsmath}
\usepackage{amsfonts}
\usepackage{amssymb}
\usepackage[mathscr]{eucal}
\usepackage{epsfig,graphics,graphicx}
\usepackage{color}
\usepackage{amsthm}
\usepackage{amstext}
\usepackage{stmaryrd}
\usepackage{graphicx}
\usepackage{setspace}
\doublespacing

\makeatletter

%%%%%%%%%%%%%%%%%%%%%%%%%%%%%% LyX specific LaTeX commands.
%% A simple dot to overcome graphicx limitations
\newcommand{\lyxdot}{.}
%%%%%%%
\newtheorem{lemma}{Lemma}

 \newcommand{\abs}[1]{\left|{#1}\right|}
 \newcommand{\av}[1]{\left\langle #1 \right\rangle}
 
  \newcommand{\br}[1]{\langle #1|}
  \newcommand{\ke}[1]{|#1\rangle}
  \newcommand{\bk}[2]{\langle #1|#2\rangle}
  \newcommand{\kb}[2]{\ke{#1}\br{#2}}
  \newcommand{\var}[2]{\langle #1,#2\rangle} 
  
  \newcommand{\al}[1]{^{(#1)}}
  \newcommand{\da}{^\dagger} 
  
  \newcommand{\pt}[1]{\left( #1 \right)}
  \newcommand{\pq}[1]{\left[ #1 \right]}
  \newcommand{\pg}[1]{\left\{ #1 \right\}} 
  
  \newcommand{\lpt}[1]{\left( #1 \right.}
  \newcommand{\lpq}[1]{\left[ #1 \right.]}
  \newcommand{\lpg}[1]{\left\{ #1 \right.}
  \newcommand{\rpt}[1]{\left. #1 \right)}
  \newcommand{\rpq}[1]{\left. #1 \right]}
  \newcommand{\rpg}[1]{\left. #1 \right\}} 


\newcommand{\td}[1]{\widetilde{#1}}

%%%%%%%%%%%%%%%%%%%%%%%%%%%%%% Textclass specific LaTeX commands.
\numberwithin{section}{chapter}
\numberwithin{equation}{section}
\numberwithin{figure}{section}

\makeatother

\usepackage{babel}

\begin{document}
\section{Implementation of Probabilistic Approximate Cloning}
In order to clone N copies of state $\ke {\psi}$ approximately we need N+1 ports for our interferometer.
This results in very complicated applications of the R-Z algorithm when N becomes large.  We therefore
demonstrate the solution on $1 \rightarrow 2$ cloning with equal prior probabilities.  If we perform this
operation as a one-shot measurement we need a 5x5 matrix and up to (more) beamsplitters.  

However we can probabilistically optimally separate the two input states using the results of the previous section,
then apply a cloning unitary to make the desired copies.  Since we know the optimal relationship
between input and ouput overlaps from the previous chapter, the first step is choosing the desired final overlap
and failure rate.  Given the final overlap we optimally deterministically transform these states into the clones.
This reduces the complexity of the problem since now we are working with a 4x4 matrix.  

\subsection{Separation Implementation}
\begin{eqnarray}
U\ke {\psi_1} = \sqrt{p_1}\ke{\phi_1}+ \sqrt{q_1}\ke{0}\\
U\ke {\psi_2} = \sqrt{p_2}\ke{\phi_2}+ \sqrt{q_2}\ke{0}
\end{eqnarray}
where the states are
\begin{eqnarray}
\ke {\psi_1} = c_1 \ke{1} + s_1 \ke{2}\\
\ke {\psi_2} = c_1 \ke{1} - s_1 \ke{2}\\
\ke {\phi_1} = c_2 \ke{1} + s_2 \ke{2}\\
\ke {\phi_2} = c_2 \ke{1} - s_2 \ke{2}\\
\end{eqnarray}
then we get the equations
\begin{eqnarray}
\br 0 U \ke {\psi_1} &=&c_1 U_{01} + s_1 U_{02} = \sqrt{q_1}\\
\br 0 U \ke {\psi_1} &=&c_1 U_{11} + s_1 U_{12} = c_2 \sqrt{p_1}\\
\br 0 U \ke {\psi_1} &=&c_1 U_{21} + s_1 U_{22} = s_2\sqrt{p_1}
\end{eqnarray}
and
\begin{eqnarray}
\br 0 U \ke {\psi_2} &=& c_1 U_{01} - s_1 U_{02} = \sqrt{q_2}\\
\br 1 U \ke {\psi_2} &=&c_1 U_{11} - s_1 U_{12} = c_2\sqrt{p_2}\\
\br 2 U \ke {\psi_2} &=& c_1 U_{21} - s_1 U_{22} = -s_2\sqrt{p_2}
\end{eqnarray}
we can solve the the two set of equations pairwise for the first six matrix elements as
\begin{equation}
{}[U]=
\begin{pmatrix}\bullet & \frac{\sqrt{q_1}+\sqrt{q_2}}{2c_1} & \frac{\sqrt{q_1}-\sqrt{q_2}}{2s_1}\\
\bullet &\frac{c_2(\sqrt{p_1}+\sqrt{p_2})}{2c_1} & \frac{c_2(\sqrt{p_1}-\sqrt{p_2})}{2s_1} \\
\bullet & \frac{s_2(\sqrt{p_1}-\sqrt{p_2})}{2c_1} & \frac{s_2(\sqrt{p_1}+\sqrt{p_2})}{2s_1} 
\end{pmatrix}
\end{equation}

  Applying the unitarity constraint
$U^\dagger U = I$ gives us nine equations for the remaining three uknown elements.  Numerical optimization
at this point is straightforward for all values of the parameters and one may follow the R-Z algorithm or Sun $\textit{et all}$ from there.
\subsubsection{Separation Implementation: Equal priors}
We give the closed form solution for the equal priors condition.  Here $p_1 = p_2$ and $q_1 = q_2$ thereby simplifying the equations.

\begin{equation}
{}[U]=
\begin{pmatrix}\bullet & \frac{\sqrt{q}}{c_1} & 0\\
\bullet &\frac{c_2\sqrt{p}}{c_1} & 0 \\
\bullet & 0 & \frac{s_2\sqrt{p}}{s_1} 
\end{pmatrix}
\end{equation}

Using the optimal sucess rate 
\begin{equation}
p = \frac{\bk{\psi_1}{\psi_2}-1}{\bk{\phi_1}{\phi_2}-1}
\end{equation}
this simplifies to
\begin{equation}
{}[U]=
\begin{pmatrix}\bullet & \frac{\sqrt{s_2^2-s_1^2}}{c_1s_2} & 0\\
\bullet &\frac{c_2s_1}{c_1s_2} & 0 \\
\bullet & 0 & 1
\end{pmatrix}
\end{equation}

implying we can choose the remaining elements to be

\begin{equation}
{}[U]=
\begin{pmatrix}\frac{c_2s_1}{c_1s_2}& \frac{\sqrt{s_2^2-s_1^2}}{c_1s_2} & 0\\
-\frac{\sqrt{s_2^2-s_1^2}}{c_1s_2} &\frac{c_2s_1}{c_1s_2} & 0 \\
0 & 0 & 1
\end{pmatrix}
\end{equation}
and only one beam splitter is needed to perform this operation.
\subsection{Deterministic Cloning Implementation}
The unitary transformation
for the second step is
\begin{eqnarray}
U\ke {\phi_1} \ke 0 = \ke{\xi_1}\ke{\xi_1}\\
U\ke {\phi_2} \ke 0 = \ke{\xi_2}\ke{\xi_2}
\end{eqnarray}
where 
$\ke {\xi_1} = c_3 \ke{1} + s_3 \ke{2}$ and
$\ke {\xi_1} = c_3 \ke{1} - s_3 \ke{2}$
Following a similar procedure we choose the basis states $ \ke {00}$,$ \ke {10}$,$ \ke {01}$, and$ \ke {11}$,
giving us the final unitary as 
\begin{equation}
{}[U]=
\begin{pmatrix} \frac{c_3^2}{c_2} &0 & 0 &\frac{s_3^2}{c_2}\\
0 &\frac{c_3 s_3}{s_2} &-\frac{c_3 s_3}{s_2}&0 \\
0 &\frac{c_3 s_3}{s_2} &\frac{c_3 s_3}{s_2}&0 \\
\frac{s_3^2}{c_2} & 0&0&\frac{c_3^2}{c_2}
\end{pmatrix}
\end{equation}
Since this last step is deterministic cloning we have the relationship between the overlaps as $ \bk {\phi_1}{\phi_2} = \bk {\xi_1}{\xi_2}^2$.  This leads to the relation $s_2^2 = 2 c_3^2 s_3^2$ and $c_3^2 = \frac{1}{2} \pm \sqrt{1-2s_2^2}$, meaning that the $M_{23}$ beamsplitter is just a 50-50 beamsplitter:
\begin{equation}
{}[U]=
\begin{pmatrix} \frac{c_3^2}{c_2} &0 & 0 &\frac{s_3^2}{c_2}\\
0 &\frac{1}{\sqrt{2}} &\frac{1}{\sqrt{2}}&0 \\
0 &\frac{1}{\sqrt{2}} &\frac{1}{\sqrt{2}}&0 \\
\frac{s_3^2}{c_2} & 0&0&\frac{c_3^2}{c_2}
\end{pmatrix}
\end{equation}
This is clearly the action of two separate beam splitters $M_{14}$ and $M_{23}$ such that

\begin{eqnarray}
{}[M_{14}]=
\begin{pmatrix}\frac{c_3^2}{c_2} &0 & 0 &\frac{s_3^2}{c_2}\\
0&1&0&0\\
0&0&1&0\\
\frac{s_3^2}{c_2} & 0&0&\frac{c_3^2}{c_2}
\end{pmatrix}
{}[M_{23}]=
\begin{pmatrix}1&0&0&0\\
0&\frac{1}{\sqrt{2}} &\frac{1}{\sqrt{2}} &0 \\
0&-\frac{1}{\sqrt{2}} & \frac{1}{\sqrt{2}}&0\\
0&0&0&1
\end{pmatrix}
\end{eqnarray}

The variables $c_2$ and $c_3$ are functions of the failure rate $Q$, the a-priori probabilities $\eta_i$ and initial overlap $c_1$.
\end{document}

