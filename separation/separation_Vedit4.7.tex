%% LyX 2.1.3 created this file.  For more info, see http://www.lyx.org/.
%% Do not edit unless you really know what you are doing.
\documentclass[twocolumn,aps,pra,eqsecnum,showpacs]{revtex4}
\usepackage[T1]{fontenc}
\usepackage[utf8]{inputenc}
\setcounter{secnumdepth}{3}
\usepackage{amsmath}
\usepackage{amssymb}
\usepackage{stmaryrd}

\makeatletter
%%%%%%%%%%%%%%%%%%%%%%%%%%%%%% Textclass specific LaTeX commands.
\@ifundefined{textcolor}{}
{%
 \definecolor{BLACK}{gray}{0}
 \definecolor{WHITE}{gray}{1}
 \definecolor{RED}{rgb}{1,0,0}
 \definecolor{GREEN}{rgb}{0,1,0}
 \definecolor{BLUE}{rgb}{0,0,1}
 \definecolor{CYAN}{cmyk}{1,0,0,0}
 \definecolor{MAGENTA}{cmyk}{0,1,0,0}
 \definecolor{YELLOW}{cmyk}{0,0,1,0}
}

%%%%%%%%%%%%%%%%%%%%%%%%%%%%%% User specified LaTeX commands.


\usepackage{bm}\usepackage{bbm}%%%%%%%%%%%%%%%%%%%%%%%%%%%%%%%%%%%%%%%%%%%%%%%%%%%%%%%%%%%%%%%%%%%%%%% %%%%%%%%%%%%%%%%%%%%%%%%%%%%%%%%%%%%%%%%%%%%%%%%%%%%%
 
 \newcommand{\abs}[1]{\left|{#1}\right|}
 \newcommand{\av}[1]{\left\langle #1 \right\rangle}
 
  \newcommand{\br}[1]{\langle #1|}
  \newcommand{\ke}[1]{|#1\rangle}
  \newcommand{\bk}[2]{\langle #1|#2\rangle}
  \newcommand{\kb}[2]{\ke{#1}\br{#2}}
  \newcommand{\var}[2]{\langle #1,#2\rangle} 
  
  \newcommand{\al}[1]{^{(#1)}}
  \newcommand{\da}{^\dagger} 
  
  \newcommand{\pt}[1]{\left( #1 \right)}
  \newcommand{\pq}[1]{\left[ #1 \right]}
  \newcommand{\pg}[1]{\left\{ #1 \right\}} 
  
  \newcommand{\lpt}[1]{\left( #1 \right.}
  \newcommand{\lpq}[1]{\left[ #1 \right.]}
  \newcommand{\lpg}[1]{\left\{ #1 \right.}
  \newcommand{\rpt}[1]{\left. #1 \right)}
  \newcommand{\rpq}[1]{\left. #1 \right]}
  \newcommand{\rpg}[1]{\left. #1 \right\}} 
  
  \newcommand{\pp}[2]{ {\mbox{\scriptsize$
  \begin{array}{c}
  #1\\
  #2
  \end{array}$} } }  
  
  

\makeatother

\begin{document}
\global\long\def\sandwich#1#2#3{ \left\langle #1\left|#2\right|#3\right\rangle }
\global\long\def\ket#1{\left|#1\right>}
\global\long\def\braket#1#2{\left\langle #1\mid#2\right\rangle }
\global\long\def\bra#1{\left\langle #1\right|}
\global\long\def\indep{\perp\!\!\!\perp}



\title{Quantum State Separation and Interpolation Between Approximate and
Probabilistic Exact Cloning}


\author{$^{1}$, $^{2}$ }


\affiliation{$^{1}$Department of Physics and Astronomy, Hunter College of the
City University of New York, 695 Park Avenue, New York, NY 10065,
USA\\
\\
 $^{2}$F\'{i}sica Teòrica: Informació i Fenòmens Quàntics, Universitat
Autònoma de Barcelona, 08193 Bellaterra (Barcelona), Spain }


\date{\today}
\begin{abstract}
%\cite{Chiribella2006,Chiribella2013,Fiurasek2002,Jimenez2010,Pomarico2011}\cite{Fiurasek2002} 
We
solve the problem of state separation of two known
pure states in the general case where they have arbitrary
a-priori probabilities. The solution emerges from a geometric formulation
of the problem. This formulation also reveals a deeper connection
between cloning and state discrimination. The results are then applied
in designing a scheme for hybrid cloning which interpolates between
approximate and probabilistic exact cloning. State separation and
hybrid cloning are generalized schemes to well established state discrimination
and cloning strategies. 
\end{abstract}
\maketitle
%%%%%%%%%%%%%%%%%%%%%%%%%%%%%%%%%%%%%%%%%%%%%%%%%%%   INTRODUCTION   %%%%%%%%%%%%%%%%%%%%%%%%%%%%%%%%%%%%%%%%%%%%%%%%%%%%%%% %%%%%%%%%%%%%%%%%%%%%%%%%%%%%%%%%%%%%%%%%%%%%%%%%%%%%%%%%%%%%%%%%%%%%%%%%%%%%%%%%%%%%%%%%%%%%%%%%%%%%%%%%%%%%%%%%%%%%%%%%%%%%%%%%%



\section{Introduction}

The strange nature of the microscopic world allows for communications (cite security) and computational(cite shor) tasks
 impossible without quantum mechanics.  We aim to investigate the ultimate limits imposed by quantum mechanics on
 communication and information processing in the more general context of probabilistic protocols. Examples are quantum cloning
 and quantum discrimination, which we describe below. 
The important common feature of these examples is that the output states must have a smaller overlap, thereby more distinguishable,
 than the input states. A process by which the overlap of the output states is smaller than that of the input states is known as state separation.
 State separation cannot be done deterministically, since it would imply that quantum states become more distinguishable.
 No result concerning general prior probabilities are known so far. For reasons analogous to those discussed
 in our PRL (symmetry,…), the equal prior solution may not be representative of general case. We provide 
the general solution here and use it to obtain the optimal interpolating scheme for cloning (perfect-max fidelity).

While it is impossible to always clone quantum states perfectly \cite{Dieks1982,Wootters1982},
this leads to some unexpected advantages for quantum systems over
classical ones in some communications protocols of practical relevance.
A common example is stronger security in cryptographic key distribution,
with recent works showing more applications \cite{Pomarico2011,Bartkiewicz2014}.
Developments in cloning, including probabilistically perfect \cite{Duan1998}
and approximate cloning \cite{Buzek1998} {[}{]} provide anchors for
better understanding quantum theory as a whole, such as the relationship
between the no-cloning and no-signaling theorems \cite{Barnum2007a},
and fundamental limits on quantum measurements . In particular, cloning’s
relationship with the fundamental limits of state discrimination will
be central to this paper.

In Exact Cloning one prepares perfect clones of the input states while
allowing for some failure rate in which no clones have been produced
and the states are discarded. In optimal unambigious discrimination
(UD) the input states are made orthogonal and hence fully distinguishable.
It was shown by Chefles and Barnett \cite{Chefles1998a} that these
strategies are a special case of a more general scheme. Both have
two outcomes: failure and success. In each strategy the overlap of
the input states is decreased, in UD the overlap becomes zer, in Exact
Cloning it is the overlap of the input states raised to the power
of the desired number of the clones to be made, $s^{N}$. State separation
unifies the two schemes as it produces states with an overlap in the
range zero to $s^{N}.$ The authors showed the results for the case
when the states are preapared with equal apriori probabilities. In
liu of our recent work on probabilistic Exact Cloning {[} {]} where
a geometric picture emerges we use similar tools to solve the more
general state separation when the input states are prepared with different
a priori probabilities. 

The results are then applied to hybrid cloning, interpolation between
probabilistic exact cloning and approximate cloning non-orthogonal
quantum states which are also prepared with different a priori probabilities.
Strategies for hybrid cloning when the incoming states are prepared
with equal apriori probabilities are derived in \cite{Chefles1999}.
In the limit of infinite number of copies the scheme reproduces that
of minimum error state discrimination with a fixed rate of inconclusive
results in \cite{Bagan2012}. 

Both state separation and hybrid cloning can be seen as generalization
schemes of well established state discrimination and cloning strategies.
In essence state separation unified two well established strategies:
The optimal UD and probabilistic Exact Cloning. Hybrid cloning unifies
approximate cloning with probabilistic exact cloning. In the limit
of producing infinite copies in unifies ME with UD. 


\section{State Separation}

We approach state separation via the Neumark formulation. The system
is embedded in a larger Hilbert space where the extra degrees of freedom
are customarily called the ancilla. Then a unitary transformation
entangles the system degrees of freedom with those of the ancilla.
The input states $\{|\psi_{1}\rangle_{s},|\psi_{2}\rangle_{s}\}$
which live in the state Hilbert space $S$ are embedded with the ancilla
$|i\rangle_{a}$ which live in the Hilbert space $A.$ Now the system
and the ancilla live in the larger Hilbert space $H=S\varotimes A.$
The incoming states in this larger Hilbert space can be written in
the product form $\{|\psi_{1}\rangle_{s}|i\rangle_{a},|\psi_{2}\rangle_{s}|i\rangle_{a}\}.$
The unitary should do the following: 

\begin{eqnarray}
U|\psi_{1}\rangle|i\rangle & = & \sqrt{p_{1}}|\phi_{1}\rangle|\alpha\rangle+\sqrt{q_{1}}|\Phi_{o}\rangle|f\rangle,\label{eq:separation1}\\
U|\psi_{2}\rangle|i\rangle & = & \sqrt{p_{2}}|\phi_{2}\rangle|\alpha\rangle+\sqrt{q_{2}}|\Phi_{o}\rangle|f\rangle,\label{eq:separation2}
\end{eqnarray}


were $\ket{\alpha}$ and $\ket f$ are orthogonal. A projective measurement
along the ancilla $\ket{\alpha}$ means that the states have succesfully
become more distinguishable with a success rate of $p_{i}$, otherwise
a measurement in the $\ket f$ space means that the process has failed
to produce more distinguishable states and the states are discarded
with a probability of $q_{i}.$ 

The inner product of (\ref{eq:separation1}) with (\ref{eq:separation2})
gives the unitarity constraint:

\begin{equation}
s=\sqrt{p_{1}p_{2}}s'+\sqrt{q_{1}q_{2}},\label{eq:constraint}
\end{equation}


where $s=|\braket{\psi_{1}}{\psi_{2}}|$ and $s'=|\braket{\phi_{1}}{\phi_{2}}|.$

When the input states are prepared with equal priors $\eta_{1}=\eta_{2},$
the solution is directly derived from Eq. (\ref{eq:constraint}) and no
further optimization is neccessary: $s=ps'+q\Rightarrow p=(1-s)/(1-s').$
For $s'=0$ the IDP limit is reached, while for $s'=s^{N}$ probabilistic
exact cloning limit is reached. 

For given a priori probabilities $\eta_{1}$, $\eta_{2}$ of our states $\ke{\psi_1}$ and $\ke{\psi_2}$, and average failure probability
\begin{equation}
Q = \eta_1 q_1 + \eta_2 q_2,
\label{failure}
\end{equation}
we wish to find the minimum value of the final overlap $s'$
as a function of the initial overlap $s$. For general a-priori probabilities,
rather than attempting to write the optimal output overlap $s'$ as an explicit function
of $s$, which would require solving a high degree polynomial equation, we will give the curve $(s,\min s')$ in parametric
form.  

We solve the problem by linearizing the constraint, (\ref{eq:constraint}), and finding the point at which it is tangent to the failure rate curve (\ref{failure}).
We choose a change of variables 
\begin{equation}
p_{1}p_{1}=t^{2}, \quad   q_{1}q_{2}=z^{2}
\end{equation}
 which linearizes the unitarity constraint
 (\ref{eq:constraint}) as 
$z=s-s't$ where $0\le t,z\le1$ and $0\le s'\le s$.
From the first of these substitutions we have $t^{2}=(1-q_{1})(1-q_{2})=1+z^{2}-q_{1}-q_{2}.$
Solving for $q_{2}$ and substituting back in the second equation
results in a quadratic equation which can be solved for $q_{1}$, giving the individual failure rates as 
\[
q_{i}=\frac{1+z^{2}-t^{2}\mp (-1)^i\sqrt{(1+z^{2}-t^{2})^{2}-4z^{2}}}{2}.
\]
Therefore, the failure rate (\ref{failure}) becomes
\[
Q=\frac{1}{2}\left(1+z^{2}-t^{2}\pm(\eta_{1}-\eta_{2})\sqrt{(1+z^{2}-t^{2})^{2}-4z^{2}}\right).
\]
We now solve for $z^{2}$. After a bit of algebra we obtain %$$%(\eta_1-\eta_2)^2\left[(1+z^2-t^2)^2-4z^2\right]=(1-2Q+z^2-t^2)^2%$$%$$%-\eta_1\eta_2\left[(1+z^2-t^2)^2-4z^2\right]=%Q^2-Q+(1-Q)z^2+Q t^2%$$%$$%\eta_1\eta_2z^4+\left[1-Q-2\eta_1\eta_2(1+t^2)\right]z^2+\eta_1\eta_2(1-t^2)^2+Q t^2-Q(1-Q)=0%$$%$$%z^2={2\eta_1\eta_2(1+t^2)-1+Q\pm\sqrt{(1-Q)^2+16\eta_1^2\eta_2^2t^2-4\eta_1\eta_2(1-Q)^2-4\eta_1\eta_2t^2}\over 2\eta_1\eta_2}%$$%$$%z^2={2\eta_1\eta_2(1+t^2)-1+Q\pm\sqrt{(1-4\eta_1\eta_2)(1-Q)^2-4\eta_1\eta_2(1-4\eta_1\eta_2)t^2}\over 2\eta_1\eta_2}%$$
\begin{widetext}
\begin{equation}
z^{2}=\frac{2\eta_{1}\eta_{2}(1+\tau)-(1-Q)+\sqrt{(1-4\eta_{1}\eta_{2})\left[(1-Q)^{2}-4\eta_{1}\eta_{2}\tau\right]}}{2\eta_{1}\eta_{2}}\equiv\zeta(\tau),
\label{zeta}
\end{equation}
\end{widetext}

where $\tau\equiv t^{2}$. %$$%z^2={Q-\left[1-2\eta_1\eta_2(1+t^2)\right]\pm\sqrt{(1-4\eta_1\eta_2)\left[(1-Q)^2-4\eta_1\eta_2t^2\right]}\over 2 \eta_1\eta_2}%$$
Since $z^{2}$ cannot be less than zero, we picked up the plus sign for
the root. %The maximum value of $t$ is then%$$%t_{\rm max}={1-Q\over 2\sqrt{\eta_1\eta_2}},%$$%and we note that %$$%1\le t_{\rm max}\qquad\mbox{\rm if $\quad Q\le 1-2\sqrt{\eta_1\eta_2}$}.%$$%For $t>t_{\rm max}$ the argument of the square root in the definition of $\zeta$ becomes negative. Thus, $\zeta(t^2)$ is singular at $t=t_{\rm max}$ and, in particular $\zeta'(t^2_{\rm max})$ does not exists ($\zeta'$ goes to $-\infty$ as $t\to t_{\rm max}$). %The values of $t$ at which $z=0$ ($\zeta=0$) are%$$%t_{0,1}=\sqrt{1-{Q\over\eta_1}},\qquad t_{0,2}=\sqrt{1-{Q\over\eta_2}}.%$$Let

We assume that $0\le\eta_{1}\le1/2$ to simplify the analysis. %Then,%$$%t_{0,1}\le t_{0,2}.%$$We
To locate the extrema of $z$, we find $dz/dt=(d\zeta/d\tau)(t/z)$.
The derivative $d\zeta/d\tau$ is immediate, given by
\[
\zeta'(\tau)=1-\frac{\sqrt{1-4\eta_{1}\eta_{2}}}{\sqrt{(1-Q)^{2}-4\eta_{1}\eta_{2}\tau}}.
\]
We find that the minimum
is located at %$$%t_{\rm th}%=\sqrt{(2\eta_1-Q)(2\eta_2-Q)\over 4\eta_1\eta_2}%=\sqrt{\left(1-{Q\over2\eta_1}\right)\left(1-{Q\over2\eta_2}\right)}, \quad   z_{\rm th}={Q\over2\sqrt{\eta_1\eta_2}}%\quad\mbox{\rm if $\quad Q\le2\eta_1$},%$$%or it is located at%$$%t'_{\rm th}%=0, \quad   z'_{\rm th}=\sqrt{Q-\eta_1\over\eta_2}%\quad\mbox{\rm if $\quad 2\eta_1\le Q$}.%$$
\[
t_{{\rm min}}=\left\{ \begin{array}{lll}
{\displaystyle \sqrt{\left(1-\frac{Q}{2\eta_{1}}\right)\left(1-\frac{Q}{2\eta_{2}}\right)},\quad} & {\rm if} & \quad0\le Q\le2\eta_{1}\\[2em]
0,\quad & {\rm if} & \quad2\eta_{1}<Q\le1.
\end{array}\right.
\]
The corresponding values of $z$ are %$$%z_{\rm min}={Q\over2\sqrt{\eta_1\eta_2}},\quad%z_{\rm min}=\sqrt{Q-\eta_1\over\eta_2}.%$$
\[
z_{{\rm min}}=\left\{ \begin{array}{lll}
{\displaystyle \frac{Q}{2\sqrt{\eta_{1}\eta_{2}}},\quad} & {\rm if} & \quad0\le Q\le2\eta_{1}\\[2em]
{\displaystyle \sqrt{\frac{Q-\eta_{1}}{\eta_{2}}},\quad} & {\rm if} & \quad2\eta_{1}<Q\le1.
\end{array}\right.
\]

Now that we have mapped out the shape of the failure rate (\ref{failure}), we want to know where it is tangent to the constraint (\ref{eq:constraint}).  We first note that
for equal priors the curve (\ref{zeta}) is simply the hyperbola $z^{2}=t^{2}+2Q-1$, which intersects the straight line 
$z=1-t$ at the point $(z,t)=(Q,1-Q)$.

 We note that this is actually a general solution of (\ref{zeta}) for any $\eta_{1}$, $\eta_{2}$. Moreover,
the straight line $z=1-t$ is tangent to~$(\ref{zeta})$ at
$(Q,1-Q)$ for any values of $\eta_{1}$, $\eta_{2}$, as can be checked
by substituting in the formula $dz/dt=(t/z)(d\zeta/d\tau)$. Since $z=1-t$ is the limiting line for the family $z=s-s't$, an obvious parametrization for the curve $(s,s')$ is obtained
as follows: {\em i}.~define 
\[
s'(t)=-\frac{dz}{dt}=-\frac{t\,\zeta'(t^{2})}{\sqrt{\zeta(t^{2})}},\qquad t_{{\rm min}}\le t\le1-Q,
\]
and next {\em ii}.~define 
\[
s(t)=z+ts'(t)=\sqrt{\zeta(t^{2})}+ts'(t),\qquad t_{{\rm min}}\le t\le1-Q.
\]
where 



For $s<z_{{\rm min}}$ is is always possible to separate the initial
states, i.e., $|\Psi_{1}\rangle$ and~$|\Psi_{2}\rangle$ can be
made orthogonal. We note that the condition $s=z_{{\rm min}}$ is
equivalent to the unambiguous discrimination result 
\[
Q=2\sqrt{\eta_{1}\eta_{2}}s,\quad Q=\eta_{1}+\eta_{2}s^{2}.
\]


The next plot is for $\eta_{1}=0.1$. As $\eta_{1}$ approaches $1/2$
the curves approach a straight line. The difference is more noticeable
for very small values of $\eta_{1}$.


\section{Hybrid Cloning}

In this section we seek to interpolate between probabilistic exact
cloning and approximate cloning machines using our results from state
separation. Exact cloning machines produce perfect clones while allowing
for some inconclusive outcomes. Approximate cloning machines produce
copies on demand which resemble the input states while maximizing
the fidelity. One can imagine a scheme where fidelity can be higher
then maximum fidelity in the approximate cloning machine while it
allowes for a fixed rate of inconlusive outcomes, $FRIO.$ This scheme
should reproduce exact cloning and approximate cloning machines by
setting $FRIO$ to $Q_{o}$ and zero respectively. Chefles and Barnett
\cite{Chefles1999} solve the problem for when the input states are
prepared with equal a priori probabilities. We extend the solution
to the more general case when the states are prepared with different
aprioris. Such a solution is possible due to our recent work in making
$n$ perfect clones from $m$ copies of one of two known pure states
with minimum failure probability in the general case where the known
states have arbitrary a-priori probabilities.

We can imagine a state dependent approximate cloner as a machine with
an input and an output port. The input states $|\psi_{i}^{M}\rangle=|\psi_{i}\rangle^{\otimes M}$,
$i=1,2$ ($m$ identical copies of either $|\psi_{1}\rangle$ or $|\psi_{2}\rangle$)
are fed through the input port for processing. The output states are
$n$ approximate clones ~$|\phi_{i}^{M}\rangle=|\phi_{i}\rangle^{\otimes M}.$
This is deterministic cloning, although imperfect, clones are generated
on demand while optimizing the global fidelity. Thus given a set $K$
of non orthogonal states $|\psi_{i}\rangle^{\otimes M},$ we wish
to produce a set $K$ of $N$ clones $|\phi_{j}\rangle^{\otimes N}$
while optimizing the global fidelity:

\begin{equation}
F_{MN}=\sum_{j=1}^{K}\eta_{j}|\langle\psi_{j}^{N}|\phi_{j}\rangle|^{2}.
\end{equation}


A unitary produces $N$ copies $|\phi_{1}\rangle$ or $|\phi_{2}\rangle,$
to resemble the original states as closely as possible:

\begin{eqnarray}
U|\psi_{1}^{M}\rangle|i\rangle & = & |\phi_{1}^{n}\rangle,\\
U|\psi_{2}^{M}\rangle|i\rangle & = & |\phi_{2}^{n}\rangle.
\end{eqnarray}


The inner product of the above two equations gives a relationship
between the input and the output states. 

\begin{eqnarray}
|\langle\psi_{1}|\psi_{2}\rangle|^{M} & = & |\langle\Phi_{1}|\Phi_{2}\rangle|^{N}\\
\cos^{M}2\theta & = & \cos^{N}(\phi_{1}+\phi_{2})\label{eq:3.4.6}
\end{eqnarray}


Where the input states are expressed as $|\psi_{1,2}^{N}\rangle=\cos\theta|1\rangle\pm\sin\theta|0\rangle$
and the clones as $|\phi_{1,2}\rangle=\cos\phi_{1}|1\rangle\pm\sin\phi_{1}|0\rangle.$

It was shown in \cite{Chefles1999} that the maximum fidelity is:

\begin{equation}
F_{MN}=\frac{1}{2}[1+\sqrt{1-4\eta_{1}\eta_{2}\sin^{2}(2\theta-(\phi_{1}+\phi_{2}))}]\label{eq:Fidelity}
\end{equation}


To see the connection with state separation expand the $\sin$ term:

$\sin(2\theta-(\phi_{1}+\phi_{2}))=\sin(2\theta)\cos(\phi_{1}+\phi_{2})+\cos(2\theta)[1-\cos^{2}(\phi_{1}+\phi_{2})]$

The overlap of the output states $\cos(\phi_{1}+\phi_{2})$ can be
viewd as state separation. In this case we are in the limit $s'=s$,
approximate cloning. To interpolate between approximate cloning and
deterministic cloning $s'$ takes on the optimal values derived in
Section II. 

\bibliographystyle{unsrt}
\bibliography{/Users/ashehu/Desktop/mendeley}

\end{document}
